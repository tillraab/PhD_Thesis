%\chapter{\mbox{Habitat preference and explorative behavior}}
\chapter{\mbox{Group dispersion and movement patterns}}
\label{Habitats}

\textbf{The literal content of this chapter has already been published as:}\\  
Till Raab, Laura Linhart, Anna Wurm, Jan Benda (2019) Dominance in Habitat Preference and Diurnal Explorative Behavior of the Weakly Electric Fish \Lepto{}. \textit{Frontiers in Integrative Neuroscience} 13:21.

\paragraph*{Contributions}
For this scientific project Till Raab designed the experiments, collected and analyzed the data, performed post-processing of electric EOD recordings, generated the displayed figures, and wrote the manuscript. 
Laura Linhart and Anna Wurm collected the data and performed post-processing of electric EOD recordings while being supervised by Till Raab. Laura Linhart additionally provided the sketches of the experimental tank displayed in \subfigref{eodtraces}{A, B}. 
Jan Benda discussed the experiments, advised data analysis, and assisted writing the manuscript. 

%\section{Abstract}
\begin{quote}
\begin{center}
\large\textbf{Abstract}
\end{center}
Electrocommunication and -localization behaviors of weakly electric fish have been
studied extensively in the lab, mostly by means of short-term observations on constrained
fish. Far less is known about their behaviors in more natural-like settings, where fish
are less constrained in space and time. We tracked individual fish in a population of
fourteen brown ghost knifefish (\Lepto{}) housed in a large 2\,m$^3$ indoor
tank based on their electric organ discharges (EOD). The tank contained four different
natural-like microhabitats (gravel, plants, isolated stones, stacked stones). In particular during the day individual fish showed preferences for specific habitats which provided appropriate shelter. Male fish with higher EOD frequencies spent more time in their preferred habitat during the day, moved more often between habitats during the night, and less often during the day in comparison to low-frequency males. Our data thus
revealed a link between dominance indicated by higher EOD frequency, territoriality, and
a more explorative personality in male \lepto{}. In females, movement activity
during both day and night correlated positively with EOD frequency. In the night, fish
of either sex moved to another habitat after about 6\,s on average. During the day, the
average transition time was also very short at about 20\,s. However, these activity phases
were interrupted by phases of inactivity that lasted on average about 20\,min during the
day, but only 3\,min in the night. The individual preference for daytime retreat sites did not reflect the frequent explorative movements at night.
\end{quote}
\hfill

\section{Introduction}
Weakly electric fish are nocturnally active. In the night, many pulse-type fish increase the
rate of their electric organ discharges (EOD) \citep{Lissmann1965,Stoddard2007}, wave-type fish emit various kinds of electrocommunication signals more frequently \citep{Zupanc2001,Henninger2018}, and gymnotids have been shown to move from deep waters up to the shore \citep{Steinbach1970}. During the day, weakly electric fish hide under submerged logs (\textit{Gymnotus}, \citealp{Westby1988}), between roots (\textit{Eigenmannia}, \citealp{Hopkins1974}), in leaf litter (\textit{Brachyhypopomus}, \citealp{Hagedorn1988}), or bury themselves in sand (\textit{Gymnorhamphichthys}, \citealp{Lissmann1965}).

EOD frequencies of the gymnotiform brown ghost knifefish \lepto{} are sexually dimorphic with males having higher EOD frequencies than females \citep{Meyer1987}. In playback experiments with restrained fish, males more frequently produced aggressive communication signals (chirps) than females \citep{ZupancMaler1993, Bastian2001} and in experiments of free swimming fish, male \lepto{} showed a higher overall chirp rate compared to females \citep{Dunlap2002,Hupe2008}. However, during courtship in the field females produced almost as many chirps as males \citep{Henninger2018}, and both sexes jammed rivals by approaching their EOD frequencies \citep{Tallarovic2002}. In competition experiments, male \lepto{} were
more likely to inhabit tubes alone, whereas females cohabited tubes more often \citep{Dunlap2002}.

Several studies suggest higher EOD frequencies in males as an indicator of dominance. Additionally, body size correlated with EOD frequency in males \citep{Triefenbach2008, Fugere2011} but not in females \citep{Dunlap2002}. Dominant males with higher EOD frequencies were more aggressive \citep{Fugere2011} and participated more in mating
\citep{Hagedorn1985, Henninger2018}. In competition experiments, males with higher EOD frequencies occupied the most preferred tubes, whereas females did not distribute according to EOD frequency \citep{Dunlap2002}. In summary, these laboratory studies suggest that male
brown ghost knifefish are territorial at their preferred retreat site during the day, and that males with higher EOD frequencies are more dominant.

Observations on aggression and dominance have previously been limited to studies in the lab in small tanks, and mostly to short observation times (e.g., \citealp{Hopkins1974, Hagedorn1985, Nelson1999, Tallarovic2005, Triefenbach2008, Hupe2008}). Recent technological advances allow for long-term observations of electric activity of these fish in the lab and in the field \citep{Henninger2018, Madhav2018}. Here, we take advantage of these methods and describe diurnal activity patterns of a community of \lepto{} competing for different
microhabitats in a large indoor tank over 10 days.

\section{Methods}
Six male and eight female \lepto{}, obtained from a tropical fish supplier, were housed in a $2.5 \times 1 \times 0.8$\,\cubic\meter{} indoor tank with a water conductivity of 320\,\micro\siemens\per\centi\meter{} at a 12\,h/12\,h light cycle. Initially, four fish inhabited the tank. Starting at day 4 we introduced two additional fish per day. Fish were selected for approximately equal size to minimize effects based on physical differences as far as possible. All fish were mature and not in breeding condition. EOD frequency is sexually dimorphic in \lepto{} \citep{Meyer1987}. We identified fish with EOD frequencies lower than 750\,Hz as females, and fish with higher EOD frequencies as males \citep{Henninger2018}. Four natural-like habitats in $60 \times 45 \times 10$\,\centi\meter\cubed{}  PVC-containers were arranged next to each other in the tank: stacked stones, quartz gravel (few millimeters diameter), isolated stones, and aquatic plants (\textit{Vallisneria spec.}) (\subfigrefb{eodtraces}{A}). Fish were fed frozen \textit{Chironomus plumosus} on the gravel habitat every day at about 8\,h after lights were switched on. Animal housing complied with national and European law and was approved by the
Regierungspr\"asidium T\"ubingen (permit no: 35/9185.46/UniT\"U). Approval by an ethics committee was not required because our study was purely observational.

We continuously recorded EODs for 10 days and nights using 16 monopolar electrodes at low-noise headstages, and digitized at 20 kHz per channel with 16 bit resolution (see \citealp{Henninger2018} for details). For each of the four habitats, two electrodes were placed at the bottom of the habitat 35\,cm apart and two electrodes 35\,cm above the respective electrodes in the habitats in the open water (\subfigrefb{eodtraces}{A, B}). Water temperature was measured once a day. During the course of the experiment, water temperature steadily dropped from 26.3\,\celsius{} to 24.8\,\celsius. Fish were identified by their specific peaks in the spectrogram of the recordings (nfft = $2^{16}$, overlap = 90\,\%) and tracked using a custom tracking algorithm comparing fundamental EOD frequency and the corresponding power pattern in the spectrograms of the different electrodes (see \citealp{Henninger2018} and \citealp{Madhav2018} for details).

Every 0.328\,ms (temporal resolution of the spectrogram), fish were assigned to habitats by means of the electrode with the largest power at the fish's EOD frequency. Based on this spatial information we analyzed how the fish occupied the habitats. For each day and night, we computed the fraction of fish in each habitat by dividing the detections within one habitat by the total number of detections on that day or night (\subfigref{eodtraces}{E}). Likewise, individual habitat preferences were computed separately based on the detections of each fish (\subfigref{habitatpref}{A}). To assess the number and composition of fish in each habitat we counted the number of males and females detected in each habitat for every time
step (\subfigref{habitatpref}{C}). The male ratio is the number of males in a habitat divided by the total number of detected fish in that habitat (\subfigref{habitatpref}{D}).

The preferred habitat of a fish was defined as the habitat where the fish spent most of the time, i.e., had the most detections, for each day and night. Relative time spent in the preferred habitat was computed as the ratio between detections in the preferred habitat and the number of detections per day or night (12\,h $\times$ 3600\,s/h $\times$ 3.05 detections per second $\approx$ 131,827 detections per 12\,h) for every day and night (\subfigref{habitatpref}{B}). The stability of individual habitat preferences was evaluated using preference change rates, i.e., the probability of a fish to change its preferred habitat from one day or night to another one, computed as the number of days (or nights) on which the fish preferred a different habitat as on the previous day (or night) divided by the number of days the fish was in the tank minus one (\subfigref{transitions}{A, C}).

Transitions of fish between habitats were characterized by the number of transitions of detections from electrodes of one habitat to electrodes from another habitat (\subfigref{transitions}{B, D}). The distributions of transition times $\Delta t$, i.e., the time spans a fish spent in one habitat between two habitat changes, were exponentially distributed (\subfigref{transitiontimes}{A}):
\begin{linenomath}
  \begin{equation}
    \label{expdist}
    p(\Delta t) = \lambda e^{-\lambda \Delta t} \; .
  \end{equation}
\end{linenomath}
The number of transitions per time (\subfigrefb{transitions}{B}) is the
transition rate. In \subfigref{transitiontimes}{B} the transition rate
$\lambda=1/\tau$ was estimated from the average transition time $\tau
= \frac{1}{n} \sum_{i=1}^n \Delta t_i$ for each fish separately for
days and nights.

The tails in the distributions of transition times dominate the activity patterns of the fish because a single long transition time implies a non-moving fish for exactly this time. During the same time, however, many more short transitions can occur. Short transition times are thus overrepresented when taking the average.  To account for this we also computed a weighted average $\overline{\Delta t_i}$, where we weighted each transition time $\Delta t_i$ by its duration $\Delta t_i$ (\subfigref{transitiontimes}{C}):
\begin{linenomath}
  \begin{equation}
    \label{weightedtime}
    \overline{\Delta t_i} = \frac{\sum_{i=1}^n \Delta t_i^2}{\sum_{i=1}^n \Delta t_i}
  \end{equation}
\end{linenomath}

Finally, we investigated if individual habitat changes were independent of each other by calculating the time differences between a fish entering a habitat and the other fish leaving the respective habitat. We compared these distributions to boot strapped distributions where entering times to a random habitat were set randomly throughout the whole recording period.

Because fish were in similar physical condition and their sexes were determined using only a hard EOD frequency cutoff at 750\,Hz we performed a sensitivity analysis for all corresponding results, i.e., additionally to the original sex assignments, all
statistics were calculated with up to $\pm 2$ males or females, where the individuals closest to the cutoff were assigned to the opposite sex.

For quantifying differences between groups, we used Cohen's $d$ for unequal group sizes:
\begin{linenomath}
  \begin{equation}
    \label{cohensd}
    d = \left| \frac{\mu_1 - \mu_2}{  \frac{n-1}{n+m-2} \sigma_1^2  +   \frac{m-1}{n+m-2} \sigma_2^2    }    \right|
  \end{equation}
\end{linenomath}
where $\mu_1$ and $\mu_2$ are the means, $\sigma_1$ and $\sigma_2$ the standard deviations, and $n$ and $m$ the group sizes, respectively.

\section{Results}

We observed the movements of six male and eight female \lepto{} between four microhabitats and the open water in a two cubic meter tank over 10 days. We tracked individual fish
based on EOD frequency and power on 16 recording electrodes (\subfigrefb{eodtraces}{C}). EOD frequency is known to be sexually dimorphic in \lepto{} \citep{Meyer1987}. Fish with an EOD
frequency above 750\,Hz are defined as males, fish below 750\,Hz as females (\subfigrefb{eodtraces}{D}, \citealp{Henninger2018}). The overall decline of EOD frequencies followed the water temperature, which decreased by 1.5\,\celsius{} over the course of the experiment. In fact, the $Q_{10}$ values computed for each fish from daily temperature
measurements and the corresponding EOD frequencies (median $Q_{10}=1.54$) were close to typical $Q_{10}$ values reported for these fish in the literature \citep{Dunlap2000, Stoeckel2014}. Additionally, circadian modulations of each fish's EOD frequency followed similar patterns and can also be best explained by periodic diurnal water temperature changes \citep{Dunlap2000}. On top of these exogenous influences, the fish actively changed
their EOD frequency, approaching and evading EOD frequencies of other fish. For example, the EOD frequencies of the males indicated in orange and blue approached each other and got
closer to the males indicated in red and green. Female fish also approached each other in their EOD frequency and even switched order (e.g., the females indicated in red and light blue at the bottom of \subfigrefb{eodtraces}{C}). In the following we do not analyze
these modulations of EOD frequency but rather focus on diurnal movement patterns.

\begin{figure*}[p]
  \showfigure{\includegraphics[width=1\linewidth]{habitats.png}\\[3ex]
    \includegraphics[width=1\linewidth]{traces_distribution}}
  \caption{\label{eodtraces} Experimental setup, EOD frequencies and
    distribution of fish over habitats. \figitem{A} Top view of the
    experimental setup with four different micro habitats (plants,
    isolated stones, gravel, stacked stones). Electrodes (red) were
    fixed in location by PVC poles positioned above the tank. Fish
    were fed on a daily basis on the gravel habitat using a custom PCV
    feeder (blue). \figitem{B} Side view of the experimental setup
    showing the electrodes positioned in two levels over the
    habitats. \figitem{C} EOD frequency traces tracked over the entire
    duration of the experiment. Individual fish are marked by the same
    color in all figures. \figitem{D} Ranges of male (red) and female
    (orange) EOD frequencies. \figitem{E} Fraction of fish detected
    within each of the five habitats for consecutive days (top) and
    nights (bottom). \figitem{F} Relative occupation of the habitats
    averaged over all days (top) and nights (bottom).}
\end{figure*}

\subsection{Habitat occupation}
The tank offered the fish four different 0.25\,m$^{2}$ habitats that contained either stacked stones, quartz gravel, isolated stones, or aquatic plants. We counted the open water above the habitats as a fifth habitat. For each time point we assigned each fish to one of
these habitats according to the electrode with the largest power at its EOD frequency.

During the days, i.e., the presumably inactive phases of the fish, most fish were located within the aquatic plants followed by the stacked stones and the isolated stones. Fish were rarely found in the gravel habitat or in the open water (\subfigrefb{eodtraces}{E, F}, top). At night, no habitat seemed to be preferred on average (\subfigrefb{eodtraces}{E, F}, bottom).

During the days, the addition of fish did not influence the distribution of fish in the habitats by a lot (\subfigrefb{eodtraces}{E}, top). The standard deviation of the fraction of fish occupying a habitat was below 6.5\,\% for all habitats. Nevertheless, the fraction of fish occupying isolated stones or gravel increased slightly throughout the experiment (Spearman's rank correlation: $r=0.76$, $p=0.005$ and $r=0.88$, $p<0.001$, respectively), whereas the occupation of the aquatic plants and the open water slightly decreased within
the 10 days (Spearman's rank correlation: $r=-0.76$, $p<0.05$ and $r=-0.65$, $p<0.05$, respectively). The occupation of the stacked stones habitat was unaffected by the increasing
fish count and did not change over the days of the experiment (Spearman's rank correlation: $r=0.37$, $p=0.30$). Consequently, the increasing total fish count led to an almost uniform increase in the number of fish occupying each habitat. None of the habitats was claimed exclusively by a dominant fish as a retreat site during the days.

In contrast, at nights the increased fish count seemed to influence the distribution of the fish over the habitats more strongly (\subfigrefb{eodtraces}{E}, bottom). The occupation of both the isolated and stacked stones habitats increased slightly during the course of the experiment (Spearman's rank correlation: $r=0.84$, $p<0.01$ and $r=0.79$, $p<0.01$, respectively), whereas the fraction of fish in the open water clearly decreased (Spearman's rank correlation: $r=-0.90$, $p<0.001$) and the occupation of the gravel habitat increased (Spearman's rank correlation: $r=0.95$, $p<0.001$). The latter could be attributed to the experimental design. Food was supplied daily at the gravel habitat and gymnotiform fish
have been shown to learn the location of food \citep{Jun2013}.

\begin{figure*}[t!]
  \showfigure{\centerline{\includegraphics[width=1\linewidth]{habitat_pref}}}
  \caption{\label{habitatpref} Habitat preference.
    \figitem{A} Relative time each individual fish spent in the
    different habitats (same color code as in
    \subfigrefb{eodtraces}{E}) averaged over all days (top) and nights
    (bottom). Males (fish IDs 1--6) are indicated in red, females
    (fish IDs 7--14) in orange. Male and female fish IDs are sorted
    according to descending EOD frequency in all figures.
    \figitem{B} For each fish and day (blue) or night (grey) the
    fraction of time the fish spent in its currently preferred
    habitat. Asterisks indicate significant differences: ***
    $p<0.001$, ** $p<0.01$, and * $p<0.05$.  \figitem{C} For each
    habitat the mean group size with standard deviation in which males
    (red) and females (orange) were found after the maximum of 14 fish
    had been reached.  \figitem{D} For each habitat the average male
    ratio with standard deviation during the day (blue) and night
    (grey) after the maximum of 14 fish had been reached. }
\end{figure*}

\subsection{Habitat preferences}
Let us now turn to the habitat preferences of individual fish (\subfigrefb{habitatpref}{A, B}). Even during the day fish did not stay at the same habitat. Male no. 2 was the only exception, which throughout the experiment stayed in the stacked stones at daytime (\subfigrefb{habitatpref}{A}, top). The preferred daytime habitat, i.e., the habitat the fish stayed the longest during the day, varied between individuals. Some fish preferred the stacked stones, whereas others preferred the isolated stones or the plants. Only male no. 6 had a slight preference for the gravel habitat. In the night, individual fish had less obvious preferences for specific habitats on average (Wilcoxon: $W=0$, $p=0.001$, \subfigrefb{habitatpref}{A}, bottom).

The fish sometimes changed their preferred habitat from one day to another (\subfigrefb{transitions}{A}). Preferred nighttime habitats were changed more often than daytime habitats (Wilcoxon: $W=3$, $p=0.005$) (\subfigrefb{transitions}{C}). The probability of changing the preferred habitat from one day to the next did not significantly correlate with EOD frequency, neither for males nor for females (Spearman's rank correlation: $p>0.2$).

In particular males stayed significantly longer in their preferred daytime habitat than in their preferred nighttime habitat (\subfigrefb{habitatpref}{B}). Furthermore, males with higher EOD frequencies spent more time in their preferred daytime habitat than low-frequency males (Spearman's rank correlation: $r=0.49$, $p<0.001$). For males at night and females no
such correlation was significant (Spearman's rank correlation: $p>0.1$).

To summarize, with the exception of male no. 2, the notion of a ``preferred habitat'' turns out to be misleading. Of course, there is always a habitat where a fish spends most time during a day or night simply by definition. However, other habitats are visited as well (\subfigrefb{habitatpref}{A, B}) and even the preferred habitat is changed within a few days (\subfigrefb{transitions}{A}).

\subsection{Group sizes and composition}

Many fish had similar habitat preferences. This should be reflected in the number of fish found in each habitat. For quantifying group sizes and compositions in the different habitats we analyzed the final 78\,h where all fourteen fish were present in the tank. The mean group size differed between the habitats (\subfigrefb{habitatpref}{C}). Significantly less fish were simultaneously detected in the open water ($1.89\pm 0.95$) than in the isolated stones ($3.39\pm 1.32$, Mann-Whitney U: $p<0.001$, Cohen's $d=1.20$), and stacked stones ($3.61\pm 1.26$, Mann-Whitney U: $p<0.001$, Cohen's $d=1.43$). Group sizes in the gravel ($5.21 \pm 2.61$) and plant habitat were significantly larger than in both stone habitats and the open water (Mann-Whitney U: $p<0.001$, Cohen’s $d$: $0.78<d<2.16$).

Interestingly, male ratios in all habitats were close to the expected 0.43 given by the overall number of six males and eight females (dashed line in \subfigrefb{habitatpref}{D}, Cohen's $d<0.24$). There was no difference in habitat preferences between the sexes. Only in the open water at night the male ratio was considerably lower than expected (Cohen's $d=0.77$).

\begin{figure*}[t]
  \showfigure{\centerline{\includegraphics[width=1\linewidth]{transitions}}}
  \caption{\label{transitions} Transitions of habitat preference and
    between habitats. \figitem{A} Probability of changing the
    preferred habitat from one day (blue) or night (grey) to the next
    for each fish. \figitem{B} Transition rates, i.e., number of
    detected transitions between habitats per 12\,h, averaged over
    days (blue) or nights (grey) with standard deviation.
    \figitem{C} Probabilities of changing preference of night habitats
    vs. preference changes of day habitats from
    \panel{A}. \figitem{D} Transition rates during the day
    vs. transition rates at night from \panel{B}.  Transition
    counts averaged over days and nights with standard deviation are
    shown for each male (red) and female (orange). Symbols in
    \panel{C}~\&~\panel{D} indicate fish ID as in \panel{B}.}
\end{figure*}

\subsection{Transitions between habitats}
Fish frequently moved between habitats (\subfigrefb{transitions}{B}). The EOD frequency of males correlated negatively with the number of transitions between habitats during the day and positively during the night (Spearman's rank correlation: $r=-0.47$, $p<0.01$ and $r=0.55$, $p<0.001$, respectively). That is, high-frequency males were more territorial during the day and more explorative at night than low-frequency males. In females, transition counts correlated positively with EOD frequency during both day and night (Spearman's rank correlation: $r=0.55$, $p<0.001$ and $r=0.45$, $p<0.01$). Therefore, females with higher EOD frequency were more active.

Both males and females switched habitats significantly more often during the night than during the day (\subfigrefb{transitions}{D}). The more stationary males were during the day, the more explorative they were at night (Spearman's rank correlation: $r=-0.49$, $p<0.001$). On the other hand, female transition counts during day and night were positively correlated ($r=0.53$, $p<0.001$). No such correlations existed for individual fish.

Transition times, i.e., the time intervals between habitat transitions, were approximately exponentially distributed (\eqnref{expdist}, \subfigrefb{transitiontimes}{A}). Such exponential distributions are generated by Poisson point processes where the probability of
an event (here a transition to another habitat) is the same for each time point and independent of previous events, like for example radioactive decay or state transitions of ion channels. There was no distinguished time scale that separated activity phases from resting phases. Transition rates (\subfigsref{transitions}{B}, \subfref{transitiontimes}{B}) were generally quite high and average to 0.1\,Hz. They were significantly larger during the night than during the day for both, males (Mann-Whitney U: $U=0$, $p<0.01$, $d =
4.05$) and females (Mann-Whitney U: $U=8$, $p<0.05$, $d = 1.58$), and were independent of sex (\subfigrefb{transitiontimes}{B}).

Averaged weighted transition times, \eqnref{weightedtime}, better capture differences on long time scales, reflecting non-moving fish. On average weighted transition times were 20\,min during the day and 3\,min at night (\subfigrefb{transitiontimes}{C}, Mann-Whitney U: males $U=0$, $p<0.01$, $d=1.41$, females $U=8$, $p<0.05$, $d=0.34$).

Transitions of individual fish were independent from other fish entering the habitat (not shown). The distribution of times between a fish entering a habitat and another fish leaving
the same habitat showed statistically significant (Kolmogorov-Smirnov test, $p<0.001$) but small differences to a distribution generated for times of a fish entering a randomly chosen habitat drawn from a uniform distribution (Cohen's $d$: $0.02<d<0.08$).

\begin{figure*}[t]
  \showfigure{\centerline{\includegraphics[width=1\linewidth]{transition_times}}}
  \caption{\label{transitiontimes} Transition times. \figitem{A}
    Probability density of transition times (time span spent
    consecutively within one habitat) during the day (blue) and the
    night (grey) for one example male (top, fish ID 4) and female
    (bottom, fish ID 10). \figitem{B} Corresponding transition rates
    obtained from fitting an exponential to the distributions of
    transition times. \figitem{C} Averaged weighted transition times
    \eqnref{weightedtime}. Asterisks indicate significant differences:
    ** $p<0.01$, and * $p<0.05$.}
\end{figure*}

\subsection{Sensitivity analysis}
Since we based the sex of the fish on EOD frequency only, we repeated all analysis for different EOD frequency thresholds separating males from females. For up to two males reassigned to females and vice versa the sex dependent results in the contexts of \subfigsref{transitions}{C, D}, \subfref{transitiontimes}{B, C} did not change. All significant levels as well as effect sizes stayed in the same range.

\section{Discussion}
We observed movement patterns and habitat preferences in a population of fourteen brown ghost knifefish, \lepto{}, in a large indoor tank over 10 days. During the day, these
nocturnal fish distributed themselves quite uniformly in habitats providing appropriate retreat sites between stones or plants. Activity at night was characterized by strong explorative movements where fish frequently changed between habitats and the open water. In male fish, high EOD frequency correlated with more territorial behavior during days and a more explorative personality at night, whereas in female fish EOD frequency was positively correlated with movement activity during both day and night.

\subsection{Nocturnal activity}
Despite the well supported common notion of weakly electric fish being nocturnally active \citep{Lissmann1965, Zupanc2001, Henninger2018}, our data show that phases of activity, as indicated by short transition times between the habitats, occurred in similar ways both at night and during the day (\subfigref{transitiontimes}{A}). There was no qualitative difference between day and night. During the day, phases of inactivity were prolonged
about ten-fold in comparison to the ones at night (\subfigref{transitiontimes}{B,\,C}).
Otherwise, activity, as quantified by transitions between habitats, occurred randomly and independently of each other. This fits well with the description of stochastic onsets of activity phases in \textit{Gymnotus} \citep{Jun2014}.

\subsection{Retreat site selection}
Selection of an appropriate retreat site has profound effects on the animal's physiological condition and fitness \citep{Rosenzweig1981, Huey1991}. All of the preferred retreat sites in our experiment offered appropriate places where fish could hide. This fits well to field observations where fish were also found hiding under submerged logs, between roots, or in leaf litter during the day \citep{Hopkins1974, Hagedorn1988, Westby1988}. Our data
demonstrates that, at least in captivity, most fish do not depend on specific retreat sites, like for example stacked stones, but rather change between many available types of microhabitats.

In small tanks in the laboratory males often compete over tubes provided for refuge \citep{Hopkins1974, Hagedorn1988, Fugere2011}. In the presence of enough tubes, male
\lepto{} preferred to occupy tubes alone, but females were sometimes found together in single tubes \citep{Dunlap2002}. Fish had clear preferences when presented with a variety of tubes of different dimensions and opacity \citep{Dunlap2002}.

In our study fish showed individual preferences for different habitats (\subfigref{habitatpref}{A}). The grass and gravel habitat accommodated the most individuals simultaneously, and the open water the least (\subfigref{habitatpref}{C}). This indicates either differences in general habitat quality or differences in the actual number of available suitable retreat sites in each of the habitats. The fraction of males found in each habitat on average did not deviate from the expectation given the total number of males and females (\subfigref{habitatpref}{D}). Thus, group composition on the scale of a whole
habitat was not influenced by the hierarchical status of individual fish. However, our experimental design did not allow to resolve group compositions on a finer spatial scale of specific retreat sites within each habitat. Our data therefore do not contradict an influence of hierarchical status on retreat site selection as reported by \citet{Dunlap2002}.

\subsection{Social dominance}
The EOD and its modulations convey information about species, sex, status and intent of individuals (e.g., \citealp{Hagedorn1985, Stamper2010, Fugere2011}). In \lepto{} EOD frequency correlates with body size \citep{Dunlap2002b, Triefenbach2003}. Furthermore, dominant males in breeding contexts in the laboratory \citep{Hagedorn1985} as well as in the field \citep{Henninger2018}, and in tube selection contexts \citep{Dunlap2002, Fugere2011} had higher EOD frequencies. We here reported a more subtle variant of dominance. Male fish
with higher EOD frequency moved less between habitats during the day and showed increased movement activity at night compared to males with lower EOD frequency. These increased
nocturnal movement activities could reflect frequent fights for dominance \citep{Tallarovic2005}, as the approaching EOD frequencies of the fish suggest (\subfigrefb{eodtraces}{C}). Contrary to the expectation of fish fighting for dominance, the time points of fish leaving a habitat were independent from fish entering the respective habitat. A closer inspection of the EOD frequency traces for communication signals like rises and chirps \citep{Zakon2002} could help classify different types of movement activities and interactions in the future \citep{Triefenbach2008}. In females, EOD frequency did not appear to be correlated with dominance \citep{Dunlap2002}. However, we found that EOD frequencies of females were positively correlated with movement activity during both day
and night. Rather than an indication of hierarchical status, EOD frequency seems to indicate individual activity personalities \citep{Sih2004}.

\subsection{Conclusion}
Many laboratory studies on the behavior of weakly electric fish focused on specific questions that were tested in temporally and spatially limited experimental settings \citep{Hopkins1974,Hagedorn1985,Nelson1999,Tallarovic2002,Hupe2008,Triefenbach2008}. Recent advances in recording techniques \citep{Madhav2018,Henninger2018} allowed us to continuously monitor a population of weakly electric fish in a large tank with a more natural-like setting for many days. In particular, we did not force the fish into specific behaviors,
but rather, extracted behavioral activity patterns from the data \citep{Gomez2014}. In this way, we revealed personality like differences in territoriality and explorative movements
\citep{Sih2004}. In both males and females these were correlated with EOD frequency, suggesting EOD frequency as an indicator for more explorative personalities in both sexes, and territoriality in males.

%\bibliographystyle{jneurosci}
%\bibliography{../journalsabbrv,../references}
